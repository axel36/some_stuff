%% start of file `template.tex'.
%% Copyright 2006-2013 Xavier Danaux (xdanaux@gmail.com).
%
% This work may be distributed and/or modified under the
% conditions of the LaTeX Project Public License version 1.3c,
% available at http://www.latex-project.org/lppl/.


\documentclass[11pt,a4paper,sans]{moderncv}        % possible options include font size ('10pt', '11pt' and '12pt'), paper size ('a4paper', 'letterpaper', 'a5paper', 'legalpaper', 'executivepaper' and 'landscape') and font family ('sans' and 'roman')

% moderncv themes
\moderncvstyle{classic}                             % style options are 'casual' (default), 'classic', 'oldstyle' and 'banking'
\moderncvcolor{blue}                               % color options 'blue' (default), 'orange', 'green', 'red', 'purple', 'grey' and 'black'
%\renewcommand{\familydefault}{\sfdefault}         % to set the default font; use '\sfdefault' for the default sans serif font, '\rmdefault' for the default roman one, or any tex font name
%\nopagenumbers{}                                  % uncomment to suppress automatic page numbering for CVs longer than one page

% character encoding
\usepackage[utf8]{inputenc}                       % if you are not using xelatex ou lualatex, replace by the encoding you are using
%\usepackage{CJKutf8}                              % if you need to use CJK to typeset your resume in Chinese, Japanese or Korean
\usepackage[english,russian]{babel}
% adjust the page margins
\usepackage[scale=0.88]{geometry}
%\setlength{\hintscolumnwidth}{3cm}                % if you want to change the width of the column with the dates
%\setlength{\makecvtitlenamewidth}{10cm}           % for the 'classic' style, if you want to force the width allocated to your name and avoid line breaks. be careful though, the length is normally calculated to avoid any overlap with your personal info; use this at your own typographical risks...

% personal data
\name{Храмов}{Алексей}
%\title{Resumé title}                               % optional, remove / comment the line if not wanted
\address{Ленинские горы дом 1}{Москва}{Россия}% optional, remove / comment the line if not wanted; the "postcode city" and and "country" arguments can be omitted or provided empty
\phone[mobile]{+7~(916)~321~14~93}                   % optional, remove / comment the line if not wanted
%\phone[fixed]{+2~(345)~678~901}                    % optional, remove / comment the line if not wanted
%\phone[fax]{+3~(456)~789~012}                      % optional, remove / comment the line if not wanted
\email{khramov@alexey.by}                               % optional, remove / comment the line if not wanted
\homepage{https://github.com/axel36}                         % optional, remove / comment the line if not wanted
%\extrainfo{additional information}                 % optional, remove / comment the line if not wanted
\photo[100pt][0.6pt]{picture}                       % optional, remove / comment the line if not wanted; '64pt' is the height the picture must be resized to, 0.4pt is the thickness of the frame around it (put it to 0pt for no frame) and 'picture' is the name of the picture file
%\quote{Some quote}                                 % optional, remove / comment the line if not wanted

% to show numerical labels in the bibliography (default is to show no labels); only useful if you make citations in your resume
%\makeatletter
%\renewcommand*{\bibliographyitemlabel}{\@biblabel{\arabic{enumiv}}}
%\makeatother
%\renewcommand*{\bibliographyitemlabel}{[\arabic{enumiv}]}% CONSIDER REPLACING THE ABOVE BY THIS

% bibliography with mutiple entries
%\usepackage{multibib}
%\newcites{book,misc}{{Books},{Others}}
%----------------------------------------------------------------------------------
%            content
%----------------------------------------------------------------------------------
\begin{document}
%\begin{CJK*}{UTF8}{gbsn}                          % to typeset your resume in Chinese using CJK
%-----       resume       ---------------------------------------------------------
%\select@language{russian}
\makecvtitle

\section{Образование}

\cventry{2014--2018}{Бакалавр}{Московский государственный университет}{Физический факультет}{}{Специальность: Математическая физика \\ Средний балл:  4.0}  % arguments 3 to 6 can be left empty
\cventry{2018--2020}{Магистр}{Московский государственный университет}{Физический факультет}{}{Специальность: Математическая физика}

\section{Дополнительное образование}
\cventry{2017-2019}{Техносфера Mail.ru group:}{}{}{}{}
\cventry{}{Разработка веб-сервисов на golang}{}{}{}{}
\cventry{}{Анализ больших объемов данных, Python, C++}{}{}{}{}
\cventry{}{Разработка на Java}{}{}{}{}
\cventry{}{Алгоритмы и структуры данных, C++}{}{}{}{}

\section{Опыт работы}
\cventry{2018}{Java разработчик}{Межбанковский технологический центр}{}{Москва}{Разработка новых и поддержкой уже существующих web-сервисов компании}
\cventry{2019}{С++/Python разработчик}{Центр инженерной физики}{}{Москва}{Разработка софта для продуктов линейки Eyepoint, написание внутренних скриптов}
\section{Навыки}
\subsection{Языки программирования}
\cvitemwithcomment{C++}{\quad C++14, STL, Concurrency, boost}{}
\cvitemwithcomment{Java}{\quad Java Core, Spring Boot, JDBC, Hibernate}{}
\cvitemwithcomment{Python}{\quad numpy, pandas, sklern}{}
\cvitemwithcomment{}{\quad Так же имею опыт работы с Golang, JavaScript, SQL, MATLAB}{}
\subsection{Знание языков}
\cvitemwithcomment{Русский}{родной}{}
\cvitemwithcomment{Английский}{перевожу технические тексты и свободно говорю}{}
\subsection{Дополнительные навыки}
\cvitemwithcomment{Права}{категория B}{}


\section{Личные Качества}
\cvitem{}{Самообучаемый, коммуникабельный, имею навыки командной работы}
\cvitem{}{В свободное время люблю играть в водное поло и играть на гитаре}                    % 'publications' is the name of a BibTeX file

% Publications from a BibTeX file using the multibib package
%\section{Publications}
%\nocitebook{book1,book2}
%\bibliographystylebook{plain}
%\bibliographybook{publications}                   % 'publications' is the name of a BibTeX file
%\nocitemisc{misc1,misc2,misc3}
%\bibliographystylemisc{plain}
%\bibliographymisc{publications}                   % 'publications' is the name of a BibTeX file

\clearpage
\end{document}


%% end of file `template.tex'.
